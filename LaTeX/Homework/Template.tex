\documentclass[a4paper,fleqn,11pt]{article}
\usepackage{BardiaHW}

\setTitle{\Large{Why We Need to Design Technology for and with Animals}}
\setAuthor{Bardia Taghavi}
\setAssignmentNumber{Seminar 1}

%  ==================================================
%  ==================================================
\begin{document}
\maketitle
\hrule
% ==================================================

\section*{\color{RosePineDawnPine}Summary}
\color{CtpMochaCrust}This seminar provides a comprehensive overview of \textbf{Animal-Computer Interaction (ACI)}, a field dedicated to designing technology from an animal-centered perspective. The core mission is to treat animals as legitimate stakeholders in the design process to enhance their welfare, support their activities, and foster better interspecies relationships.

The speaker illustrates these principles through various case studies. These include creating a \textit{wearer-centered} framework to design less intrusive tracking devices for wildlife; developing sensor-based systems that capture the nuanced, spontaneous reactions of bio-detection dogs to avoid forcing them into simplistic signals; and designing accessible controls that allow assistance dogs to operate doors and switches in human-centric environments. The research highlights an iterative design process where direct feedback from animal users—such as a dog choosing to use its paw instead of its snout—informs critical improvements to a product's usability and physical form.

Beyond functional tools, the seminar covers technology for enrichment, such as interactive acoustic systems for elephants and entire habitats for tigers that encourage meaningful movement. The presentation concludes by emphasizing the strong ethical framework of ACI, which prioritizes animal consent and welfare, framing the practice as a form of multi-species justice that supports an animal's capability to flourish.

\end{document}

